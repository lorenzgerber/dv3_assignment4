\documentclass[a4paper,11pt,twoside]{article}
%\documentclass[a4paper,11pt,twoside,se]{article}

\usepackage{UmUStudentReport}
\usepackage{verbatim}   % Multi-line comments using \begin{comment}
\usepackage{courier}    % Nicer fonts are used. (not necessary)
\usepackage{pslatex}    % Also nicer fonts. (not necessary)
\usepackage[pdftex]{graphicx}   % allows including pdf figures
\usepackage{listings}
\usepackage{pgf-umlcd}
\usepackage{amsmath}
\usepackage{amsfonts}
\usepackage{amssymb}
%\usepackage{lmodern}   % Optional fonts. (not necessary)
%\usepackage{tabularx}
%\usepackage{microtype} % Provides some typographic improvements over default settings
%\usepackage{placeins}  % For aligning images with \FloatBarrier
%\usepackage{booktabs}  % For nice-looking tables
%\usepackage{titlesec}  % More granular control of sections.

% DOCUMENT INFO
% =============
\department{Department of Computing Science}
\coursename{Language and Computation 7.5 p}
\coursecode{5DV162}
\title{Assignment 4}
\author{Lorenz Gerber ({\tt{dv15lgr@cs.umu.se}}, {\tt{lozger03@student.umu.se})}}
\date{2016-09-20}
%\revisiondate{2016-01-18}
\instructor{Henrik Björklund}


% DOCUMENT SETTINGS
% =================
\bibliographystyle{plain}
%\bibliographystyle{ieee}
\pagestyle{fancy}
\raggedbottom
\setcounter{secnumdepth}{2}
\setcounter{tocdepth}{2}
%\graphicspath{{images/}}   %Path for images

\usepackage{float}
\floatstyle{ruled}
\newfloat{listing}{thp}{lop}
\floatname{listing}{Listing}



% DEFINES
% =======
%\newcommand{\mycommand}{<latex code>}

% DOCUMENT
% ========
\begin{document}
\lstset{language=C}
\maketitle
\thispagestyle{empty}
\newpage
%\tableofcontents
%\thispagestyle{empty}
%\newpage

\clearpage
\pagenumbering{arabic}

\section*{Problem 1, BLUE-RED-GREEN}

\section*{Problem 2, Reduction $ALL_{CFG}$ to $EQ_{CFG}$}
If we can decide wether $L(E) \subseteq L(G)$ we could choose $L(G) =
\Sigma^{*}$ and decide it's universality. But we know that $ALL_{CFG}$ is
not decidable. Hence $EQ_{CFG}$ is also not decidable. 

\section*{Problem 3, Proof closure under compliment for reduction}
This can be concluded from the logical equivalence: $p \leftrightarrow q$ is the same as $\neg p \leftrightarrow \neg q$. So $p = q \in A$ and $q = f(w) \in B$. $A /leq_{m} B$ means there is a total computable $f$ such that for all $w$, $w \in A \leftrightarrow f(w) \in B$. By the argument above, this is the same as $w \notin A \leftrightarrow f(w) \notin B$, or equivalently $w \in \bar{A} \leftrightarrow(w) \in \bar{B}$.

\section*{Problem 4, Asymptotic Problems}
\subsection*{a)}
Find $c$:\\
$\lim_{n \to \infty} \frac{f(n)}{f(g)}+1$ where \\
$\lim_{n \to \infty}(\frac{5n^{4}+3n^{2}+4}{n^{4}})+1 = \lim_{n \to \infty}(5+ \frac{3}{n^{2}}+\frac{4}{n^{4}})+1 \implies c = 6$\\
find $n_{0}$ by trial: $n_{0} = 2$ \\

\subsection*{b)}
First find by trial a $n$ where $2^{n} < n!$. Then proof by induction that this holds for all $n > n_{0}$.\\
Choose $n_{0} = 4$\\
Suppose thatn when $n = k(k \geq 4)$, we have that $k! > 2^{k}$.\\
Now, prove that $(k + 1)! > 2^{k+1}$ when $n = (k + 1)(k \geq 4)$.\\
$(k + 1)! = (k + 1)k! > (k + 1)2^{k}$.\\
This implies $(k+1)! > 2^{k} \cdot 2$\\
Therefore, $(k + 1)! > 2^{k+1}$\\
Hence, $n! > 2^{n}$ for all integers $n \geq 4$\\

\subsection*{c)} 
Assume $2^{2n} = O(2^{n})$\\
then $lim_{n \to \infty} \frac{f(n)}{g(n)}+1$\\
but $lim_{n \to \infty} \frac{2^{2n}}{2^{n}} + 1 = \infty$
hence, $2^{2n} \neq O(2^{n})$\\

\subsection*{d)}
For every constant $a$ there is a constant $c$ such that $(n + a)^{3} = O(n^{3})$ 





\addcontentsline{toc}{section}{\refname}
\bibliography{references}

\end{document}
 
